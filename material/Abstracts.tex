\documentclass{article}
\usepackage{amsmath,amssymb}
\usepackage[top=20truemm,bottom=25truemm,left=15truemm,right=15truemm]{geometry}
\begin{document}
\title{New developments in Kobayashi-Hitchin correspondence and Higgs bundles \\
\medskip
Schedule Table and Abstracts}
%\author{Natsuo Miyatake}
\date{}
\maketitle

\begin{table}[htbp]
\begin{center}
\begin{tabular}{cccccc} \hline \hline
&8/5 &8/6 &8/7 &8/8 &8/9 \\ \hline
10:00-11:00&& Miyatake&Chen (9:00-10:00)&Brunebarbe&McCleerey\\ \hline
11:30-12:30&Moraru& Pingali&Wu (10:20-11:00) &Ou&Schaffhauser\\ \hline
14:30-15:30 &Mochizuki & Yamakawa &Ono (11:20-12:00)&Gomez \\ \hline
16:00-17:00  &Li & Huang &Fujioka (12:20-13:00)& Logares\\ \hline\hline
%17:20-18:00 &Miyatake  &&& \\ \hline\hline
\end{tabular}
\end{center}
\end{table}

\begin{itemize}
\item {\bf Yohan Brunebarbe} (Universit\'e de Bordeaux)

{\it Universal coverings of complex algebraic varieties}

What complex analytic spaces can be obtained as the universal covering of a complex algebraic variety? Motivated by this question, Shafarevich asked whether the universal covering of any smooth projective variety X is necessarily holomorphically convex. In other words, is there a proper holomorphic map from the universal covering of X to a Stein analytic space? Although still open, Shafarevich's question has received partial positive answers, for example when the fundamental group of X admits a faithful complex linear representation (Eyssidieux-Kaztarkov-Pantev-Ramachandran). In my talk, I will discuss the generalisation of Shafarevich's question to non-compact algebraic varieties. This is joint work with Ben Bakker and Jacob Tsimerman.


\item {\bf Xuemiao Chen} (University of Waterloo) 

{\it Tangent cones of admissible Hermitian-Yang-Mills connections }

Admissible Hermitian-Yang-Mills (HYM) connections are singular HYM connections with natural geometric bounds. In higher dimensional gauge theory, they naturally appear on the boundary of the moduli space of Hermitian-Yang-Mills connections over Kaehler manifolds. A fundamental problem was to study the uniqueness of the tangent cones of admissible HYM connections. I will explain joint work with Song Sun which confirms the uniqueness by showing that the tangent cones are algebraic invariants of the underlying reflexive sheaf.

\item {\bf Hitoshi Fujioka} (Research Institute for Mathematical Sciences, Kyoto University)

{\it Harmonic metrics for Higgs bundles of rank 3 in the Hitchin section}

Given a tuple of holomorphic differentials on a Riemann surface, one can
define a Higgs bundle in the Hitchin section and a natural symmetric
pairing of the Higgs bundle. We discuss whether a Higgs bundle of rank 3
in the Hitchin section has a compatible harmonic metric when the spectral
curve is a 2-sheeted branched covering of the Riemann surface. In
particular, we give a condition under which Higgs bundles in the Hitchin
section on $C$ or $C^\ast$; have compatible harmonic metrics.


\item {\bf Tomas Gomez} (Instituto de Ciencias Matematicas)

{\it Harder-Narasimhan filtration for principal bundles using polynomial
stability}

For vector bundles there are two notions of stability: slope
stability (defined by Mumford and Takemoto) and polynomial stability
(defined by Maruyama and Gieseker). Each of them produce a different
Harder-Narasimhan filtration. For principal bundles we have the slope
notion of stability (defined by Ramanathan) and the corresponding
Harder-Narasimhan filtration. We will present a Harder-Narasimhan
filtration for the polynomial notion of stability for principal bundles,
using the new "beyond GIT" theory developed by Alper, Halpern-Leistner and
Heinloth (joint work with A. Fernandez Herrero and A. Zamora).


\item {\bf Pengfei Huang} (Max-Planck-Institut for Mathematics in the Science)

{\it Filtered Stokes local systems and a wild nonabelian Hodge correspondence}

Analogous to the role of filtered local systems in Simpson's tame nonabelian Hodge correspondence, filtered Stokes local systems are the appropriate topological objects in wild nonabelian Hodge correspondence. In this talk, we will introduce filtered Stokes local systems and demonstrate a purely algebro-geometric construction of their moduli spaces. As an application, we will derive a wild nonabelian Hodge correspondence. This approach is applicable to general reductive groups. Based on joint works with Hao Sun.

\item {\bf Qiongling Li} (Chern Institute of Mathematics, Nankai University)

{\it  Index and total curvature of minimal surfaces in noncompact symmetric spaces and wild harmonic bundles}

We prove two main theorems about equivariant minimal surfaces in arbitrary nonpositively curved symmetric spaces extending classical results on minimal surfaces in Euclidean space. First, we show that a complete equivariant branched immersed minimal surface in a nonpositively curved symmetric space of finite total curvature must be of finite Morse index. It is a generalization of the theorem by Fischer-Colbrie, Gulliver-Lawson, and Nayatani for complete minimal surfaces in Euclidean space. Secondly, we show that a complete equivariant minimal surface in a nonpositively curved symmetric space is of finite total curvature if and only if it arises from a wild harmonic bundle over a compact Riemann surface with finite punctures. Moreover, we deduce the Jorge-Meeks type formula of the total curvature and show it is an integer multiple of $2\pi/N$ for $N$ only depending on the symmetric space. It is a generalization of the theorem by Chern-Osserman for complete minimal surfaces in Euclidean n-space. This is joint work with Takuro Mochizuki (RIMS).


\item {\bf Marina Logares} (Complutense University of Madrid)

{\it Moduli spaces and ACIS}

Integrable systems, known for their rich mathematical structure, are fundamental in both classical and modern physics. This talk explores the definition and significance of algebraically completely integrable systems (ACIS), and the moduli space of Higgs bundles. Emphasizing its applications, including the Grothendieck-Springer resolution, we will illustrate the profound connections to representation theory, algebraic geometry, and beyond. This talk is based on joint work with I. Biswas, T. G\'omez, J. Martens, A. Pe\'on-Nieto and S.Szab\'o.


\item {\bf Nicholas McCleerey} (Purdue University)

{\it Geodesic Rays in the Donaldson--Uhlenbeck--Yau Theorem}

We give new proofs of two implications in the Donaldson--Uhlenbeck--Yau theorem. Our proofs are based on geodesic rays of Hermitian metrics, inspired by recent work on the Yau--Tian--Donaldson conjecture. This is joint work with Mattias Jonsson and Sanal Shivaprasad.


\item {\bf Natsuo Miyatake} (Mathematical Science Center for Co-creative Society, Tohoku University)

{\it Harmonic metrics, subharmonic functions, and entropy}

Let $X$ be a Riemann surface and $K_X\rightarrow X$ the canonical bundle. For each integer $r\geq 2$, each $q\in H^0(K_X^r)$, and each choice of the square root $K_X^{1/2}$ of the canonical bundle, we canonically obtain a Higgs bundle, called a cyclic Higgs bundle. In this talk, I will introduce some new concepts regarding cyclic Higgs bundles. First, I will generalize the cyclic Higgs bundle concept to be associated with a multivalued section $q_N^{r/N}$, where $q_N$ is a holomorphic section of $K_X^N\rightarrow X$. One of the motivations for introducing the above notion is to establish the theory of the asymptotic behavior of ``random harmonic metrics" on cyclic Higgs bundles and their convergence. Second, I will introduce a generalization of the Hitchin equation for cyclic Higgs bundles associated with a nonnegative singular Hermitian metric on $K_X\rightarrow X$, obtained by infinitely increasing the degree of multivalence of the multivalued Higgs field. Third, I will propose a new concept, which I call entropy, that quantifies the degree of mutual misalignment of each component of diagonal harmonic metrics on cyclic Higgs bundles. In introducing the concept of entropy, I was motivated by the domination estimate for harmonic metrics on cyclic Higgs bundles established by Dai-Li and Li-Mochizuki. Finally, I will further generalize the generalized Hitchin equation for cyclic Higgs bundles to complex higher-dimensional K\"ahler manifolds. I will present the results obtained so far regarding these new concepts and discuss their potential for further development.

\newpage
\item {\bf Takuro Mochizuki} (Research Institute for Mathematical Sciences, Kyoto University)

{\it Asymptotic behaviour of the Hitchin metric of the moduli space of Higgs bundles}

The moduli space of stable Higgs bundles of degree $0$ is equipped with the hyperk\"ahler metric, called the Hitchin metric. On the locus where the Hitchin fibration is smooth, there is also the hyperk\"ahler metric called the semi-flat metric, associated with the algebraic integrable systems with the Hitchin section. As predicted by Gaiotto, Moore and Neitzke, the difference between the metrics along the curve $(E,t\theta)$ $(t\geq 1)$ decays in an exponential way. There still remains the problem of refining the exponential order to the predicted order.

In this talk, we shall discuss the issue in the rank two case. We would like to explain that we can achieve it in the symmetric case, and what obscures in the non-symmetric case.


\item {\bf Ruxandra Moraru} (University of Waterloo)

{\it The ``wild" Vafa-Witten equations and $T$-branes}

The Vafa-Witten equations are a higher-dimensional analogue of the Hitchin equations on compact Riemann surfaces for oriented four-manifolds. On a compact complex surface $X$, their solutions are polystable Higgs bundles (with Higgs fields $\phi$ taking values in a holomorphic vector bundle $E$ twisted by the canonical bundle of the surface); $T$-branes are solutions whose Higgs fields are ``non-abelian". In this talk, we consider a more general set of equations whose solutions are ``wild" Vafa-Witten pairs $(E,\phi: E \rightarrow E \otimes L)$, where $L$ is any line bundle on $X$. In particular, we give necessary conditions for the existence of non-trivial solutions of the ``wild" Vafa-Witten equations and describe some of their moduli.


\item {\bf Takashi Ono} (Osaka University)

{\it Deformation of pairs of complex manifolds and Higgs bundles and their Kuranishi spaces}

In the first half of this talk, I would like to introduce the deformation problem of pairs of complex manifolds and Higgs bundles. I introduce the Differential Graded Lie algebra (DGLA) which governs this deformation problem. In the latter half, I show that this DGLA decomposes in a certain sense when the underlying manifold is a compact K\"ahler manifold and the Higgs bundle is polystable and its Chern classes are zero. We apply this result to study the structure of the Kuranishi spaces of the pair.

\item {\bf Wenhao Ou} (Academy of Mathematics and System Science, Chinese Academy of Science)

{\it Orbifold modification of complex analytic varieties}

We prove that if $X$ is a compact complex analytic variety, which has quotient singularities in codimension 2, then there is a projective bimeromorphic morphism $f\colon Y\to X$, such that $Y$ has quotient singularities, and that the indeterminacy locus of $f^{-1}$ has codimension at least 3 in $X$. As an application, we deduce the Bogomolov-Gieseker inequality on orbifold Chern classes for stable reflexive coherent sheaves on compact Kaehler varieties which have quotient singularities in codimension 2.


\item {\bf Vamsi Pritham Pingali} (Indian Institute of Science)

{\it Gravitating vortices and cosmic strings}

The gravitating vortex (GV) equations on a compact Riemann surface arise as a dimensional reduction of the Kaehler-Yang-Mills equations. A special case of the GV equations includes the equations governing the (as of now, hypothetical) cosmic strings. I shall describe existence, uniqueness, and algebro-geometric obstructions to existence to these equations. This talk is based on joint work with M. Garcia-Fernandez, L. Alvarez-Consul, O. Garcia-Prada, and Chengjian Yao.

\item {\bf Florent Schaffhauser} (Heidelberg University)

{\it Higgs bundles for nonconstant groups and applications}

Torsors under nonconstant groups are a natural generalisation of principal bundles. If we start from a flat group bundle, we can study flat torsors under that group bundle and ask for a Higgs counterpart to such objects. However, the precise formulation of the Kobayashi-Hitchin correspondence in that setting requires some care, in particular for the notion of stability. In this talk, we present an elementary approach to this problem, using equivariant techniques. The applications we have in mind concern mostly twisted local systems over orbifolds and real algebraic curves, for which the equivariant approach is indeed sufficient.

%Torsors under nonconstant groups are a natural generalisation of principal bundles. If we start from a flat group bundle, we can study flat torsors and ask for a Higgs counterpart to such objects. However, the precise formulation of the Kobayashi-Hitchin correspondence in that setting requires some care, in particular for the notion of stability. In this talk, we present an elementary approach to this problem, using equivariant techniques. The applications we have in mind concern mostly twisted local systems over orbifolds and real algebraic curves, for which the equivariant approach is indeed sufficient.


%\item {\bf Ryosuke Takahashi} (Tohoku University)

%{\it $J$-equation, LYZ equation and a Kobayashi-Hitchin-type correspondence on semistable vector bundles}

%We introduce the J-equation on higher rank holomorphic vector bundles with an application to the Leung-Yau-Zaslow equation through the small volume limit. On semistable bundles over smooth projective surfaces, we provide a necessary and sufficient condition for the solvability of the J-equation in an asymptotic setup. Our result can be thought of as a perturbed version of the Kobayashi-Hitchin correspondence.


\item {\bf Di Wu} (Nanjing University of Science and Technology)

{\it Canonical metrics on vector bundles and applications}

In this talk, we shall introduce our recent progress in Kobayashi-Hitchin correspondences for Hermitian-Einstein metrics and harmonic metrics on vector bundles. Secondly, we will also discuss its further applications on Higgs bundles in Corlette-Simpson correspondence and on a semilinear elliptic partial differential equation in prescribed Chern scalar curvature problem. 

\item {\bf Daisuke Yamakawa} (Tokyo University of Science)

{\it Fourier-Laplace transform of rational connections}

The Fourier-Laplace transform is known to give interesting symmetries of moduli spaces of meromorphic connections on the Riemann sphere. In this talk I will review various properties of the Fourier-Laplace transform, and discuss the Fourier-Laplace transform of meromorphic connections with orthogonal/symplectic structure group.

\end{itemize}



\end{document}




